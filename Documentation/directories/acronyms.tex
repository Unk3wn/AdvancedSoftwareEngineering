%======================================================================
%                                              
%	    /\                                        
%	   /  \   ___ _ __ ___  _ __  _   _ _ __ ___  
%	  / /\ \ / __| '__/ _ \| '_ \| | | | '_ ` _ \ 
%	 / ____ \ (__| | | (_) | | | | |_| | | | | | |
%	/_/    \_\___|_|  \___/|_| |_|\__, |_| |_| |_|
% 	                              __/ |          
%	                              |___/             
%
%----------------------------------------------------------------------
% Author : Nico Holzhäuser
% Descripton : Cover Page
%======================================================================

\chapter*{Abkürzungsverzeichnis} % chapter*{..} --> keine Nummer, kein "Kapitel"
						         % Nicht ins Inhaltsverzeichnis
\addcontentsline{toc}{chapter}{Abkürzungsverzeichnis}   % Damit das doch ins Inhaltsverzeichnis kommt

% Hier werden die Abkürzungen definiert
% 1-10 nur damit der Abstand default auf diese Länge gesetzt wird
\begin{acronym}[12345678910]
  	%\acro{Name}{Darstellung der Abkürzung}{Langform der Abkürzung}
  	\acro{API}[API]{Application Programming Interface}
  	\acro{DDD}[DDD]{Domain Driven Design}
	\acro{DHBW}[DHBW]{Duale Hochschule Baden-Württemberg}
	\acro{UML}[UML]{Unified Modeling Language}
	\acro{UC}[UC]{Use Case}


	% Folgendes benutzen, wenn der Plural einer Abk. benöigt wird
	% \newacroplural{Name}{Darstellung der Abkürzung}{Langform der Abkürzung}
	\newacroplural{Abk}[Abk-en]{Abkürzungen}

	% Wenn neicht benutzt, erscheint diese Abk. nicht in der Liste
	\acro{NUA}[]{Not Used Acronym}
	
\end{acronym}
