%%%%%%%%%%%%%%%%%%%%%%%%%%%%%%%%%%%%%%%%%%%%%%%%%%%%%%%%%%%%%%%%%%%%%%%%%%%%%%%
%% Descr:       Vorlage für Berichte der DHBW-Karlsruhe
%% Author:      Prof. Dr. Jürgen Vollmer, vollmer@dhbw-karlsruhe.de
%% Modified :	Nico Holzhäuser, TINF19B4
%% -*- coding: utf-8 -*-
%%%%%%%%%%%%%%%%%%%%%%%%%%%%%%%%%%%%%%%%%%%%%%%%%%%%%%%%%%%%%%%%%%%%%%%%%%%%%%%

\chapter{Domain Driven Design}
	

	\section{Ubiquitous Language}
	Die \hk{ubiquitous language} ist ein Begriff, den \citeauthor{evans2004ddd} in seinem Buch \citetitle{evans2004ddd} eingeführt hat. Dieser Begriffs beschreibt die allgegenwärtige Sprache, die von Softwareentwickler*innen und Fachexpert*innen gemeinsam gesprochen wird \cite{ubiquitousLanguage.entwicklerDE}. Sie soll der Basis für die Entwicklung des Softwaremodells sein.
	
		\subsection{Analyse der Ubiquitous Language}
		In diesem Projekt ist die \hk{ubiquitous language} relativ einfach gehalten, da die Fachdomäne überschaubar ist. Aufgrund dieser beschränkten Problemdomäne sollten keine schwerwiegenden Kommunikationsprobleme in der gemeinsamen Sprache auftreten. Jedoch ist es auch hier wichtig, bestimmte Thematiken und Objekttypen exakt zu spezifizieren um Problemen direkt vor deren Entstehung entgegenzuwirken.
		
		\begin{table}[ht]
			\begin{adjustbox}{width=1\textwidth}
				\begin{tabular}{|l|c|}
					\hline
					\multicolumn{2}{|c|}{Ubiquitous Language} \\
					\hline
					Ubiquitous Language			&		\hk{normale} Definition \\
					\hline
					\hline
					Country						& 		Entität für das Herkunftsland einer Flasche \\
					\hline
					Manufacturer				& 		Hersteller / Abfüller der Flasche \\
					\hline
					Whiskeyflasche				&		Kleinste Entität des Systems. Einfach nur eine schöne, meist teure, Flasche Whiskey \\
					\hline
					Serie						&		0 .. n Whiskeyflaschen bilden zusammen eine Serie. Diese haben meist einen gemeinsamen Faktor, wie z.B. die Herkunft \\
					\hline
				\end{tabular}
			\end{adjustbox}
		\end{table}
	
			\subsubsection{Verbindungen zwischen den Objekten \cite{projektantrag.holzhaeuser}}
			Die Grundlage sollen einzelne Whiskeyflaschen-Objekte bilden. Diese werden durch ihre \textbf{Herkunft, das Alter, den Kaufpreis, die Sorte und den aktuellen Status} definiert. Hierbei müssen bei einer Objektanlage zwingend ein \textbf{Label} und eine \textbf{Manufacture} definiert werden. Diese Objekte können als \textbf{einzelne Flasche oder als Teil einer Serie} angelegt werden. Hierbei soll ein Objekt nur Teil \textbf{einer oder keiner} Serie sein können. Außerdem sollen Objekte als \textcolor{gray}{'Favoriten'},\textcolor{gray}{'Unverkäufliche'} bzw. als \textcolor{gray}{im Angebot zum Verkauf} gekennzeichnet werden können. Hierbei können \textcolor{red}{favorisierte} sowie \textcolor{red}{unverkäufliche} Flaschen nicht in den Status \textcolor{gray}{'zum Verkauf'} wechseln, ohne vorher die entsprechenden Kennzeichnungen zu entfernen.
			
			Eine Serie besteht aus einer Anzahl an Objekten > 1. Hierbei können verschiedene Flaschen zu einer Serie zusammengefasst werden. Eine Serie benötigt immer zwingend eine Bezeichnung und eine Menge an zugehörigen Objekten. Sind alle Objekte in einer Serie mit dem Status 'zum Verkauf' gekennzeichnet, kann die ganze Serie '**zum Verkauf**' angeboten werden. Werden Objekte gelöscht, so soll die Serie automatisch angepasst, und bei einem Inhalt von weniger als zwei Flaschen gelöscht werden.
			
			Objekte sowie können des weiteren gelöscht, sowie verändert werden.
	
		\subsection{Probleme bei der Ubiquitous Language \& deren Lösung \cite{ubiquitousLanguage.medium}}
		\begin{itemize}
			\item Übersetzen von komplexen Sachinhalten in einfache Sprache
			\begin{itemize}
				\item[] → Durch Wortdefinitionen eliminiert
			\end{itemize}
			\item Unterschiedliche Bezeichnungen für ein und das Selbe
			\begin{itemize}
				\item[] → Begriffe aus dem Definitionspool verwenden
			\end{itemize}
			\item Abstraktion der technischen Sachverhalte für die Domain-Experten
			\begin{itemize}
				\item[] → Klare und deutliche Definitionen im Team und ständige Weiterentwicklung der Ubiquitous Language
			\end{itemize}
			\item Keine Rücksichtnahme der Domain-Experten bei der Entwicklung des technischen Modells
			\begin{itemize}
				\item[] → Mit einbeziehen aller Parteien für das bestmögliche Ergebnis
			\end{itemize}
		\end{itemize}
	
	\section{Analyse und Begründung der verwendeten Muster}
		
		\subsection{Analyse}
		Eine komplette Übersicht des \acl{DDD} ist im Anhang in \cref{a.1.ddd} beigefügt.
			
			\subsubsection{Value Objects \cite{valueObjects.medium}} \label{1.va}
			Das Value Object ist ein in der Softwareentwicklung häufig eingesetztes Entwurfsmuster. Wertobjekte (engl. \hk{Value Objects}) sind unveränderbare Objekte, die einen speziellen Wert repräsentieren. Soll der Wert geändert werden, so muss ein neues Wertobjekt erzeugt werden.
			
			\subsubsection{Entities \cite{domainModell.microsoft}} \label{1.ent}
			Entitäten stellen im Domänenmodell Objekte da, welche nicht nur durch die darin enthaltenen Attribute definiert, sondern primär durch ihre Identität, Kontinuität und Persistenz im Laufe der Zeit definiert werden. Entitäten sind im Domänenmodell sehr wichtig, da sie die Basis eines Modells darstellen.
			
			\subsubsection{Aggregates \cite{domainModell.microsoft}} \label{1.aggregates}
			Ein Aggregat umfasst mindestens eine Entität: den sogenannten Aggregatstamm, der auch als Stammentität oder primäre Entität bezeichnet wird. Darüber hinaus kann es über mehrere untergeordnete Entitäten und Wertobjekte verfügen, wobei alle Entitäten und Objekte zusammenarbeiten, um erforderliche Verhaltensweisen und Transaktionen zu implementieren. \cref{a.1.aggregate}
			
			\subsubsection{Repositories \cite{repository.medium}}
			Ein Repository vermittelt zwischen der Domänen- und der Datenzuordnungsebene mithilfe einer sammlungsähnlichen Schnittstelle für den Zugriff auf Domänenobjekte. Hierbei kann die Anwendung einfach vorhandene Entitäten aus der Persistenzebene laden, speichern, updaten oder löschen.
			
			\subsubsection{Domain Service \cite{domainService.gorodinski}}
			Eric Evans beschreibt in seinem Buch \citetitle{evans2004ddd} einen Domänen Service wie folgt : 
			\begin{displayquote}
				When a significant process or transformation in the domain is not a natural responsibility of an ENTITY or VALUE OBJECT, add an operation to the model as standalone interface declared as a SERVICE. Define the interface in terms of the language of the model and make sure the operation name is part of the UBIQUITOUS LANGUAGE. Make the SERVICE stateless.
			\end{displayquote}
			Hierbei ist grob gesagt, dass ein signifikanter Prozess oder eine Transformation, die über die Verantwortung einer Entität hinaus geht, als eine eigenständige Operation des Modells als Interface mit dem Namen \hk{Service} hinzugefügt werden muss. Somit beschreibt ein Service einen Prozess, der über die Verantwortung einer Entität oder eines Wertobjektes hinaus geht.
		
		\subsection{Begründung}
		
			\subsubsection{Value Objects}
			In diesem Projekt sollen Wertobjekte dazu dienen, die Übertragung zwischen Backend und Frontend zu gewährleisten. Meist werden Sie für Read-,Update- oder Modifizierungsaufrufe verwendet. Hierbei sind diese \hk{DTO} Objekte unveränderlich und haben meist die Inhalte einer Entität. So können Entitäten in einer Serialisierten Form ausgetauscht werden.
			
			\subsubsection{Entities}
			In diesem Projekt sind vier Entitäten vorhanden, die die Basis des Domänenmodells bilden :
			\begin{itemize}
				\item \class{Country} - Land
				\item \class{Manufacturer} - Hersteller/Abfüller
				\item \class{Bottle} - Flasche
				\item \class{Serie} - Flaschenserie
			\end{itemize}
			Diese werden durch ihre Identität selbst beschrieben, werden persistiert und haben veränderliche Attribute.
			
			\subsubsection{Aggregates}
			Aggregate sind in diesem Projekt im eigentlichen Sinne nicht vorhanden. Jedoch sind nach der Definition in \cref{1.aggregates} auch einzelne Entitäten ein Aggregatstamm und somit ein Aggregat im jeweiligen Bereich. Folgen wir diesem Prinzip gibt es in diesem Projekt \textbf{vier} Aggregate.
			
			\subsubsection{Repositories}
			In diesem Projekt sind Repositories im Backend zu finden. Hierbei werden sie durch Implementierungen des Interfaces \interface{JpaRepository<Object,PrimaryKey>} umgesetzt. Mit der Hilfe von Hibernate wird so die Persistierungsschicht mit der Applikationsschicht verbunden und Methoden geschaffen, um Objekte aus der Datenbasis zu filtern.
			
			\subsubsection{Domain Service}
			Services finden in diesem Projekt ebenfalls im Backend ihre Anwendung. Hierbei wird die Logik in sogenannte Services ausgelagert, die bestimmte Entitäten oder Beziehungen anhand eines \ac{UC} bearbeiten. Hierbei sind in Spring Boot oft die \ac{API} Aufrufe einzelne gekapselte Logiken, welche durch einen entsprechenden Serviceaufruf umgesetzt werden. \\ 
			Des weiteren sind Security oder transaktionale Komponenten denkbare Services für eine Spring Anwendung.