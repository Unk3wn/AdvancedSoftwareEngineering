%%%%%%%%%%%%%%%%%%%%%%%%%%%%%%%%%%%%%%%%%%%%%%%%%%%%%%%%%%%%%%%%%%%%%%%%%%%%%%%
%% Descr:       Vorlage für Berichte der DHBW-Karlsruhe
%% Author:      Prof. Dr. Jürgen Vollmer, vollmer@dhbw-karlsruhe.de
%% Modified :	Nico Holzhäuser, TINF19B4
%% -*- coding: utf-8 -*-
%%%%%%%%%%%%%%%%%%%%%%%%%%%%%%%%%%%%%%%%%%%%%%%%%%%%%%%%%%%%%%%%%%%%%%%%%%%%%%%

\chapter{Refactoring}

	\section{Codesmell 1 - RSPEC-1128 - Unnecessary imports should be removed \cite{codeSmell1.sonar}}
	Dieser Codesmell wird in der Regel \hk{RSPEC - 1128} behandelt. Er befasst sich damit, dass ungenutzte Imports (dt. Einbettungen) aus den entsprechenden Klassen entfernt werden sollen.
	\subsubsection{Begründung}
	Dieser Fehler soll verhindern, dass der Entwickler durch zu viele Imports in einer Klasse verwirrt wird.
		
		\subsection{Fix}
		Entfernen der ungenutzten Imports. Dies kann oft durch die Option \hk{Code Refactoring} der IDE automatisch übernommen werden.
		
		\subsection{Projektbezug}
		Hierbei wurde der Fehler erkannt (\cref{a.4.backend.codesmell1.before}) und im gleichen durch die Refacrtoring Funktion (\cref{a.4.backend.codesmell1.fix}) der \ac{IDE} beseitigt (\cref{a.4.backend.codesmell1.after}).
		

	\section{Codesmell 2 - RSPEC-5411 - Boxed \hk{Boolean} should be avoided in boolean expressions \cite{codeSmell2.sonar}}
	Dieser Codesmell wird in der Regel \hk{RSPEC - 5411} behandelt. Er befasst sich damit, dass Boolean Variablen nicht direkt als Entscheidungskriterium einer Verzweigung verwendet werden sollen.
		\subsection{Begründung}
		Dieser Fehler verhindert, dass bei einem Nullwert der Boolean Variable eine \textit{NullPointerException} geworfen wird. Hierbei wird die Robustheit der Anwendung gesteigert.
		
		\subsection{Fix}
		Um diesen Fehler zu vermeiden sollten der Boolean Ausdruck in einen Boolean Wrapper geschrieben werden. Hierbei muss die nicht gültige Version 
		
		\begin{lstlisting}[language=java,caption={Non Complient Version - Codesmell 1},gobble=11]
			Boolean b = getBoolean();
			if (b) {  // Noncompliant, it will throw NPE when b == null
				foo();
			} else {
				bar();
			}
		\end{lstlisting}
	
		wie folgt abgewandelt werden.
	
		\begin{lstlisting}[language=java,caption={Complient Version - Codesmell 1},gobble=11]
			Boolean b = getBoolean();
			if (Boolean.TRUE.equals(b)) {
				foo();
			} else {
				bar();  // will be invoked for both b == false and b == null
			}
		\end{lstlisting}
	
		\subsection{Projektbezug}
		Hierbei wurde der Fehler erkannt (\cref{a.4.backend.codesmell2.before}) und im gleichen Zug eine Lösung für den Code Smell implementiert (\cref{a.4.backend.codesmell2.after}).
		
	
	\section{Code Smell 3 - RSPEC-1186 - Functions should not be empty \cite{codeSmell3.sonar}}
	Dieser Codesmell wird in der Regel \hk{RSPEC - 1186} behandelt. Er befasst sich damit, dass es keine leeren Functionen in Typescript geben sollte.
		\subsection{Begründung}
		Leere Funktionen sind allgemein zu entfernen, um ungewollte Verhaltensmuster in der Produktivanwendung zu verhindern. Des weiteren könnte, wenn diese Methode nicht unterstützt oder später entwickelt wird, eine Fehlermeldung geworfen werden. 
		
		\subsection{Fix}
		Hierbei sollten leere Funktionen entweder entfernt oder,
		\begin{lstlisting}[language=java,caption={Non Complient Version - Codesmell 3},gobble=11]
			function foo() {
			}
			
			var foo = () => {};
		\end{lstlisting}
		
		wie folgt mit einem Kommentar abgewandelt werden, warum diese Methode einen leeren Rumpf aufweist.
		
		\begin{lstlisting}[language=java,caption={Complient Version - Codesmell 3},gobble=11]
			function foo() {
				// This is intentional
			}
			
			var foo = () => {
				do_something();
			};
		\end{lstlisting}
		
		\subsection{Projektbezug}
		Hierbei wurde der Fehler erkannt (\cref{a.4.frontend.codesmell1.before}) und im gleichen Zug eine Lösung für den Code Smell implementiert (\cref{a.4.frontend.codesmell1.after}).
